\section{Primera consigna: Capturando tráfico}

\subsection{Implementación de \emph{tool} de escucha pasiva}
Utilizamos el ejemplo provisto por la documentación de \emph{Scapy} y lo visto en clase. 
El mismo se encuentra modificado para que se almacenen todos los paquetes Ethernet de la red local durante cierto tiempo. \\

Para correr el programa se debe ejecutar el comando: \\ \\ \centerline{python src/tp.py T} \\ \\ donde T es el intervalo de tiempo en segundos 
que correrá la herramienta.

\subsection{Análisis de entropía de la red}

\subsubsection{Fuente de información}
Definimos $S$ como fuente de información para el trabajo a los protocolos que se pueden identificar a través del campo $type$ 
proveniente de cada paquete. Es decir, se han tomado como símbolos a cada uno de los protocolos de los $n$ paquetes que se han escuchado en cada red. Nos encontramos con que a muchos paquetes no se les puede extraer el campo $type$, lo cual se debe a que
el campo correspondiente a ese valor representa otra cosa: la longitud del paquete. Lo que hicimos con dichos paquetes fue
ignorarlos, ya que nos pareció que la información que pudieran llegar a proveer no sería de gran importancia para el trabajo. \\

Además, para cada captura, se ha definido la fuente de información $S1$ con el objetivo de distinguir los nodos (host) de la red. La fuente $S1$
está basada únicamente en paquetes $ARP$. Definimos como fuentes de información para el trabajo a las direcciones IP fuente ($F_{src}$) y destino ($F_{dst}$) de los paquetes ARP. 
Tomando como símbolos $s \in S1$ a cada una de las $n$ direcciones IP de la red que se han escuchado, cada símbolo se corresponde a un host o 
entidad de red (Broadcast, por ejemplo).
Notamos importante que combinando éstas fuentes de información nos permite representar cualquier flujo de mensajes $ARP$ a través de una sucesión
de pares de símbolos [$s_i$, $s_j$] ($i,j \in [1\ldots n]$).
Es importante notar que tomaremos a $F_{src}$ y a $F_{dst}$ como \emph{fuentes de memoria nula}, ya que el análisis que estamos haciendo es a 
nivel $ARP$, en el que la búsqueda de determinados hosts suele ocurrir a partir de necesitar su ubicación $MAC$, usualmente por refresh automático
de la tabla $ARP$ de cada dispositivo, independiente de los mensajes anteriores observados.
Sería muy distinto si se estuvieran observando protocolos orientados a conexión o a transferencia de datos por ejemplo, en lugar
de pequeñas dosis de información posicional para mantener el funcionamiento de la red.\\

Consecuentemente poseemos todos los parámetros necesarios para poder calcular y analizar la entropía de cada una de las fuentes,
ya que la cantidad de información que brinda cada símbolo depende inversa-logarítmicamente de su probabilidad de ocurrencia.\\

\subsubsection{Implementacion de herramientas para experimentos y cálculo de entropía y probabilidades}
Para llevar a cabo los experimentos requeridos por el enunciado se implemento una herramienta utilizando \emph{Python} y la biblioteca de 
análisis de tráfico \emph{scapy}.

%Aclarar que 2048 es IP, 2054 es ARP, etc... para el que vea lo que devuelve el tp tenga idea. Esto para la fuente S

El sistema de monitorio de tráfico captura todos los paquetes en una red y los guarda en un arreglo para su posterior análisis.
Luego, se guardan los ips destino, origen más el tipo de operacion (\emph{Who-Has} ó \emph{Is-At}). También nos guardamos las $MAC$ que interactúan
durante la escucha. Esto puede observarse en el archivo \emph{tp.py}.\\

%poner como se ejecuta el programa: sudo python tp.py y el tiempo de sniff con el que vamos a entregar: ejemplo 20 minutos (1200 segundos).

Luego de cada escucha son calculadas la cantidad de paquetes de cada host distinguiendo si son destino o fuente y el tipo de operación realizada.
A partir de estos datos podemos calcular su probabilidad y poder distinguir los nodos de la red en el momento estudiado.\\

Cuánto más dejemos correr el \emph{script} más certero será el modelado de la red, pudiendo determinar qué entidades aparecen 
muy seguido y brindan poca información, así como qué ocurrencias son anomalías estadísticas y brindan una mayor información. 
Es esperable que la entropía calculada paso a paso tienda a la entropía de la fuente de información.

\subsection{Capturas sobre distintas LAN}
Para la ejercitación en este trabajo corrimos los scripts desarrollados. Por un lado, analizamos
qué ocurre en una red doméstica y por otro lado qué es lo que ocurre en la red wifi del Abasto Shopping y en las redes de la Facultad: 
LaboratoriosDC y AulasDC.\\

Hemos tomado dos enfoques distintos para encarar estas redes, de acuerdo a la información que conocíamos de antemano de cada una. Para la red 
doméstca, partimos sabiendo a qué dispositivo se corresponde cada dirección IP y cómo se relacionan los mismos, permitiéndonos ver cómo se reflejaba 
esto en el traspaso de paquetes. Para las redes del Abasto y de la Facultad, analizamos el flujo de datos desde una posición completamente ajena 
a la organización de la red; no sabemos qué dispositivo tiene cada IP ni tenemos suposiciones previas al respecto. 




