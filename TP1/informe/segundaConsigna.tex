\section{Segunda consigna: gráficos y análisis}

\subsection{Red con información previa}
A continuación detallamos la información previa que tenemos de la red:\\
\begin{table}[htb]
\begin{center}
\begin{tabular}{|l|l|}
\hline
IP & Host \\
\hline \hline
192.168.1.1 & Router \\ \hline
192.168.1.3 & Celular (wifi) \\ \hline
192.168.1.6 & Computadora (wifi)  \\ \hline
192.168.1.10 & Computadora (wifi) \\ \hline
192.168.1.36 & Computadora (cable ethernet) \\ \hline
\end{tabular}
\caption{Información previa - Red Doméstica}
\label{tabla informacion}
\end{center}
\end{table}

%192.168.1.1 -> ip del router
%192.168.1.3 -> ip de celular (wifi)
%192.168.1.6 -> ip de mi computadora (wifi)
%192.168.1.36 -> computadora conectada por cable ethernet
%192.168.1.10 -> computadora conectada (wifi) con Windows 7

Presentamos los datos obtenidos de \emph{escuchar} una red doméstica por un período de 4 horas.\\
\begin{figure}[h!]
\centering
\includegraphics[width=\textwidth]{./img/probaS_casa.jpg}
\caption{Protocolos Distinguidos - Fuente S - Hogar}
\end{figure}
\newpage

\begin{figure}[h!]
\centering
\includegraphics[width=\textwidth]{./img/proba_src_casa.jpg}
\caption{Probabilidad Source - Fuente S1(ARP) - Hogar}
\end{figure}

\begin{figure}[h!]
\centering
\includegraphics[width=\textwidth]{./img/proba_dst_casa.jpg}
\caption{Probabilidad Destination - Fuente S1(ARP) - Hogar}
\end{figure}
\newpage

\begin{figure}[h!]
\centering
\includegraphics[width=\textwidth]{./img/arp_whoHas_casa.jpg}
\caption{Source/Destination paquetes ARP - Operación: Who-Has - Hogar}
\end{figure}

\begin{figure}[h!]
\centering
\includegraphics[width=\textwidth]{./img/arp_isAt_casa.jpg}
\caption{Source/Destination paquetes ARP - Operación: Is-At - Hogar}
\end{figure}


\subsection{Redes sin información previa}
A continuación se muestra la cantidad de paquetes ARP por host en escala logarítmica. 
En el momento de realizar los gráficos, notamos que contabamos con una gran cantidad de hosts con muy baja probabilidad. 
Para mejorar la claridad y permitir apreciar los resultados a gran escala, decidimos filtrar estos casos y expresarlos como $Otros$ ya que 
no vimos valor en mostrar nodos con un 0,05 de probabilidad.\\

\subsubsection{Red Wifi: laboratorioDC}
\begin{figure}[h!]
\centering
\includegraphics[width=\textwidth]{./img/proba_src_laboDC.jpg}
\caption{Probabilidad Source - Fuente S1(ARP) - laboratoriosDC}
\end{figure}

\begin{figure}[h!]
\centering
\includegraphics[width=\textwidth]{./img/proba_dst_laboDC.jpg}
\caption{Probabilidad Destination - Fuente S1(ARP) - laboratoriosDC}
\end{figure}
\newpage

\begin{figure}[h!]
\centering
\includegraphics[width=\textwidth]{./img/arp_whoHas_laboDC.jpg}
\caption{Source/Destination paquetes ARP - Operación: Who-Has - laboratoriosDC}
\end{figure}

\begin{figure}[h!]
\centering
\includegraphics[width=\textwidth]{./img/arp_isAt_laboDC.jpg}
\caption{Source/Destination paquetes ARP - Operación: Is-At - laboratoriosDC}
\end{figure}


\newpage
\subsubsection{Red Wifi: aulasDC}
\begin{figure}[h!]
\centering
\includegraphics[width=\textwidth]{./img/probaS_aulasDC.jpg}
\caption{Protocolos Distinguidos - Fuente S - aulasDC}
\end{figure}
\newpage

\begin{figure}[h!]
\centering
\includegraphics[width=\textwidth]{./img/proba_src_aulasDC.jpg}
\caption{Probabilidad Source - Fuente S1(ARP) - aulasDC}
\end{figure}

\begin{figure}[h!]
\centering
\includegraphics[width=\textwidth]{./img/proba_dst_aulasDC.jpg}
\caption{Probabilidad Destination - Fuente S1(ARP) - aulasDC}
\end{figure}
\newpage

\begin{figure}[h!]
\centering
\includegraphics[width=\textwidth]{./img/arp_whoHas_aulasDC.jpg}
\caption{Source/Destination paquetes ARP - Operación: Who-Has - aulasDC}
\end{figure}

\begin{figure}[h!]
\centering
\includegraphics[width=\textwidth]{./img/arp_isAt_aulasDC.jpg}
\caption{Source/Destination paquetes ARP - Operación: Is-At - aulasDC}
\end{figure}


\newpage
\subsubsection{Red Wifi: A DEFINIR}