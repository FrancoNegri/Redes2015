\section{Conclusiones}
Las experimentaciones anteriormente realizadas nos permitieron deducir, a través de la captura
de paquetes $ARP$ durante un intervalo aceptable de tiempo información sobre tanto la topología de la red como las
interacciones entre los distintos dispositivos conectados a ella. Quedan claros los límites abarcados por el protocolo,
no siendo útil para realizar análisis más avanzados acerca de las comunicaciones y la transferencia de datos entre las
diferentes entidades (sólo resuelve cómo cada host sabe a dónde mandar los datos). Cabe destacar que el
estudio de $ARP$ permite ver de cerca la relación entre el nivel de red y el nivel de enlace, y la importancia de que
estos niveles puedan trabajar en conjunto para, al menos poder detectar la ubicación física de los diversos nodos de
que conforman una red LAN y generar así la base para la comunicacióon entre los diversos dispositivos.\\

Nos resulta notorio ver que la cantidad de símbolos involucrados no necesariamente altera la entropía de una fuente
de información en gran medida, si no que importa más la distribución de los mismos. Analizando la fórmula vemos que
es deducible, ya que la mayor entropía se produce cuando todos los símbolos son equiprobables, y termina representando
la cantidad de bits que serán necesarios para identicar a todos los posibles símbolos (log2(cantSimbolos)). La cantidad
de símbolos nos marca la entropía máxima, pero la distribución estadística de la aparición de símbolos determina en
que parte del espectro se encuentra la fuente de información.\\

Por último, creemos que este trabajo práctico nos deja la mente abierta a las nuevas posibilidades que se nos
presentan a la hora de analizar el tráfico de una red, tanto a nivel de protocolos como de algoritmos de ruteo y simil. 
No sirvió para lograr visualizar mediante grafos y gráficos todas las interacciones que existen para la construcción de las conexiones 
que utilizamos diariamente.