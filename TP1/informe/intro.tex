\section{Introducción Teórica}

\subsection{Fuentes de información}

En el presente trabajo práctico utilizaremos el trasfondo teórico visto en clase para modelar lo que
observamos de una determinada red como fuente de información. Para entender este concepto,
es necesario primero definir o explayarnos sobre algunas ideas subyacentes.\\

\emph{Definición:} Sea $E$ un suceso que puede presentarse con probabilidad $P(E)$. Cuando $E$ ocurre,
decimos que hemos recibido $I(E)$ unidades de información.
$I(E)$ se calcula de acuerdo a la siguiente fórmula:

\begin{center}
\begin{math}
I(E) = \log{\frac{1}{P(E)}} = -\log{P(E)}
\end{math}
\end{center}

La base en la que calculemos el logaritmo determina la denominación de la unidad de información (por ejemplo, \emph{bit} al utilizar logaritmo de base 2).

Extraemos de la fórmula que cuanto más escasa sea la posibilidad de que ocurra un evento, mayor es la cantidad de 
información que nos brinda, independientemente de en qué escala lo estemos midiendo. En este trabajo, utilizaremos
\emph{bits}, ya que es la unidad de información que manejan los dispositivos digitales (se asume sabido que la funcionalidad
de dispositivos digitales se basa dígitos con dos posibles estados determinados por umbrales de voltaje: 1 y 0, por lo que al observar
estos dígitos binarios, tenemos dos posibles alternativas equiprobables, siendo la cantidad de información obtenida al
especificar cualquiera de ellas de \emph{1 bit} de acuerdo a la ecuación de información mostrada).\\

\emph{Definición:} Dado un alfabeto finito y fijo $S$, y una probabilidad fija para cada uno de los simbolos, definimos como \emph{fuente de información} a cualquier entidad que emita símbolos de dicho alfabeto de acuerdo a las probabilidades antes establecidas. Al
observar una \emph{fuente de información} emitiendo símbolos, lo que obtenemos es una secuencia de dichos simbolos,
emitidos sucesivamente.\\ 

\emph{Definición:} Para este informe supondremos que trabajamos con \emph{fuentes de memoria nula}, esto es, que los símbolos
que son emitidos son estadísticamente independientes entre sí (cuál es el $n$-avo símbolo no depende de ninguno/s de los $n-1$ símbolos emitidos anteriormente), siendo $P(s)$ la probabilidad de que la fuente emita ese símbolo en cada iteración (entendiendo
\emph{iteración} como el proceso a través del cual la \emph{fuente} emite un símbolo).\\

Tenemos entonces hasta ahora una \emph{fuente de memoria nula} $F$, un alfabeto fuente $S$ de $n$ símbolos y una lista
de probabilidades $P(s_1) \ldots P(s_n)$. Utilizando la definición de información previamente suministrada, interpretamos como 
sucesos a las ocurrencias de los simbolos del alfabeto $S$, teniendo que la información que nos da cada ocurrencia del símbolo
$s_i$ ($i\in[1\ldots n]$) puede ser representada de la siguiente forma:

\begin{center}
\begin{math}
I(s_i) = \log_2{\frac{1}{P(s_i)}} = -\log_2{P(s_i)}
\end{math}
\end{center}

\emph{Definición:} Definimos como \emph{entropía} a la cantidad media de información por símbolo de una fuente.
Como sabemos calcular la cantidad de información que brinda la ocurrencia de cada símbolo, y tenemos las correspondientes
probabilidades de aparición, podemos calcular la información media suministrada por la fuente $F$ de la siguiente manera:

\begin{center}
\begin{math}
H(S)= \sum_{i=1}^{n} P(s_i) I(s_i) = \sum_{i=1}^{n} P(s_i) \log_2{\frac{1}{P(s_i)}}$ bits$ 
\end{math}
\end{center}

\emph{Propiedad 1:} La entropía es no negativa y se anula si y sólo si un único símbolo puede aparecer (no hay más de un símbolo
o el resto tiene $P(s_i)=0$.\\

\emph{Propiedad 2:} La entropía se maximiza cuando la aparición de todos los símbolos es equiprobable, luego:

\begin{center}
\begin{math}
H(S)= \sum_{i=1}^{n} P(s_i) \log_2{\frac{1}{P(s_i)}} =  n (1/n \log_2{n}) = \log_2{n}$ bits$ 
\end{math}
\end{center}

$H(S)$ representa también una medida de incertidumbre con respecto a la fuente $S$. Ahondaremos más al respecto de
este concepto en la correspondiente experimentación y análisis, de acuerdo a lo visto en la teórica y los aspectos a destacar explicitados en el enunciado.\\

\subsection{ARP}

También nos dedicaremos a estudiar y analizar algunas 
redes de computadoras a través de la observación de paquetes del tipo $ARP$ que
interceptaremos del medio, de acuerdo a lo especificado por el enunciado.\\

Cuando decimos un ``paquete'' $ARP$, nos estamos refiriendo a un paquete de datos
que interpretamos de acuerdo al protocolo $ARP$ 
(del inglés \textit{Address Resolution Protocol}). Este protocolo de la capa de enlace
es el que se encarga de encontrar la dirección Ethernet(MAC-Hardware) que se corresponde
con cada dirección IP que le interesa a cada host de una red. Es lo que permite que
las computadoras se comuniquen en su red local a través de direcciones IP, independientes
de las que aparecen en Internet.\\

Cuando un host quiere comunicarse con otro, necesita saber a qué dispositivo enviar
la información. Necesita entonces, dado un IP, saber a qué dirección física Ethernet
debe enviar sus paquetes para que lleguen a destino. El procedimiento que establece
el protocolo $ARP$ para lograr eso es el siguiente:

\begin{itemize}
 \item El host fuente envia un paquete ARP a la dirección física \textit{Broadcast}, preguntando por
 quién posee (tipo de mensaje \textit{Who-Has}) la dirección física que se corresponde
 con el IP con el que se desea comunicar.
 \item Algún host que posea dicha información (por ejemplo, el propietario de dicha IP)
 le responde al host (no a \textit{Broadcast}) cuál es la dirección MAC que está buscando.
\end{itemize}

Los mensajes $ARP$ se caracterizan por tener dos campos de direcciones de nivel superior
(fuente y destino) que se quieren comunicar (en nuestro caso IPv4), y dos campos direcciones
de capa de enlace: la del host fuente y la destino. En el pedido
de dirección (ARP Request, \textit{Who-Has}), el host que desea comunicarse completa
con sus datos los campos de fuente, y coloca en el campo destino IPv4 la dirección de
nivel superior con la que desea comunicarse, colocando como destino MAC a la dirección
\textit{Broadcast} de la red. Cuando se le responde, el host que reconoció
la dirección IP contesta (ARP Response, \textit{Is-At}) poniendo sus datos en los 
campos de fuente (contestando efectivamente con su dirección MAC) y los datos del host
que emitió el request en los campos de destino.\\

A rasgos generales, el protocolo $ARP$ responde a una necesidad muy básica que surge
al intentar comunicar entidades a través de un medio. Para poder enviar la información
al destino que se desea, es necesario saber dónde se encuentra este, cosa que se puede
lograr preguntándole a todas las entidades en nuestro dominio \textit{Broadcast} si
saben cómo comunicarse con el destino que se está buscando. Cuando alguna entidad lo
sepa, responderá acordemente.\\

Para el correcto funcionamiento de las redes en las que se utiliza este protocolo, es
claro que se requiere algún tipo de \textit{memoria} en donde guardar la información
que se va recopilando, así como alguna política de \textit{refresco (refresh)} para
renovar los datos que se tienen. En la materia vimos que las interfaces de red mantienen tablas que mapean direcciones
IPv4 con direcciones MAC, cada host sabiendo a qué dispositivo físico debe mandar sus
datos para que lleguen a la dirección IPv4 con la que se quiere comunicar. También vimos
que periódicamente los hosts vuelven a buscar a los elementos que ya tenían mapeados,
para enterarse de nuevas asignaciones de direcciones IP y actualizar sus datos en
consecuencia.

\subsection{Herramientas}

Utilizaremos para nuestros análisis las herramientas provistas y sugeridas por la cátedra,
en particular $Scapy$ y $Wireshark$. Las modificaciones realizadas a lo provisto se
especifican en las siguientes páginas.