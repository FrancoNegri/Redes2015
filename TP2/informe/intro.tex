\section{Introducción}

En este trabajo práctico hemos realizado nuestra propia implementación de la herramienta $traceroute$.
Esta herramienta permite conocer la ruta de los paquetes en una conexión $end$ $to$ $end$. 
En nuestra implementación enviamos paquetes $ICMP$ e incrementamos gradualmente el valor de $TTL$ empezando con un valor fijo igual a 1 (uno). 
Utilizamos esta herramienta para realizar distintas experimentaciones, entre ellas conocer las rutas que atraviesan los paquetes hasta 
llegar a cuatro universidades localizadas en distintos puntos de la Tierra y calcular los $RTTs$ relativos de los distintos hops.
Una vez obtenida esta información se utilizará para detectar enlaces submarinos entre continentes. Nos hemos basado que ante grandes variaciones
de $RTT$ podríamos estar en presencia de un enlace submarino.

\subsubsection{Objetivos utilizados}
Se utilizarán como objetivos las siguientes universidades. Las mismas están ubicadas en diferentes partes del mundo:
\begin{enumerate}
\item Universidad París Descartes - Francia (www.univ-paris5.fr) (IP: 193.51.86.16)
\item Universidad de Nigeria - Nigeria (www.unn.edu.ng) (IP: 162.144.89.24)
\item Universidad de Hong Kong - China (www.ust.hk) (IP: 143.89.14.2)
\item Universidad de Sydney - Australia (www.sydney.edu.au) (IP: 129.78.5.11)
\end{enumerate}


