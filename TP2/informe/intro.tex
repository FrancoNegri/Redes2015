\section{Introducción}

En el siguiente trabajo práctico hemos realizado nuestra propia implementación de la herramienta $traceroute$.
Esta herramienta permite conocer la ruta de los paquetes en una conexión $end$ $to$ $end$. 
En nuestra implementación enviamos paquetes $ICMP$ e incrementamos gradualmente el valor de $TTL$ empezando con un valor fijo igual a 1 (uno). 
Utilizamos la herramienta para realizar distintas experimentaciones, entre ellas conocer las rutas que atraviesan los paquetes hasta que 
llegan a cuatro universidades localizadas en distintos puntos de la Tierra, y calcular los $RTTs$ promedio de los distintos saltos.
Una vez obtenida esta información se utilizó para detectar enlaces submarinos entre continentes. Nos hemos basado que ante grandes variaciones
de $RTT$ podríamos estar en presencia de un enlace submarino.

\subsubsection{Host Destino}
Se han utilizado sitios web de universidades dado que es más probable que los servidores web se encuentren en el propio país.
Las universidades están ubicadas en otros continentes. Antes de los análisis hemos hecho un $ping$ para comprobar 
el estado de la comunicación de nuestro host con el host destino. De esta manera no sólo hemos comprobado la conexión sino que además hemos obtenido 
la $IP$ del host destino.\\

Se utilizarán como objetivos las siguientes universidades:
\begin{enumerate}
\item Universidad París Descartes - Francia (www.univ-paris5.fr) (IP: 193.51.86.16)
\item Universidad de Musku - Rusia (www.msu.ru) (IP: 188.44.50.103)
\item Universidad de Hong Kong - China (www.ust.hk) (IP: 143.89.14.2)
\item Universidad de Sydney - Australia (www.sydney.edu.au) (IP: 129.78.5.11)
\end{enumerate}


