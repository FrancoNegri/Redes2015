\section{Conclusiones}
Una vez finalizados los experimentos nos encontramos con el problema de que las herramientas de geolocalización no son del todo correctas. 
Dichas herramientas utilizadas, en algunos casos, nos dieron como resultado ubicaciones muy inexactas o ilógicas tomando como base los $RTT$ promedio obtenidos. 
Donde seguramente debería existir un enlace submarino las $IPs$ de ambos lados de la ruta nos dieron dentro de un mismo país o en ubicaciones muy cercanas. 
Relacionado con este tema no queríamos dejar de mencionar que existen routers que resultan invicibles al $traceroute$. Estos routers no contestan los paquetes 
$echo-request$ y por lo tanto no podemos averiguar su $IP$ ni sus tiempos de respuesta. Estos routers que no brindan información perjudican el análisis de la 
información y la correcta ubicación de las rutas.\\

Todos los experimentos hacia las universidades fueron realizados bajo el proveedor de Internet Fibertel. Siempre han pasado por Estados Unidos.
El caso de la universidad de Sydney nos generó curiosidad dado que se encuentran ambos host (el nuestro y el de la universidad) en el hemisferio sur. 
Sin embargo la ruta de internet pasa por Estados Unidos provocando que se extiendan los tiempos de respuesta. Si existiese algún proveedor de internet que no 
pase siempre por Estados Unidos se podrían generar distintas rutas que podrían ser favorables o no según el destino. Por ejemplo, es favorable ir hasta Estados Unidos 
si queremos llegar a Hong Kong pero no lo es si queremos ir a Australia.\\

Hemos podidos visualizar que los paquetes no van directo al destino sino que pasan por varios servidores antes de llegar al host final. 
También hemos notado que no siempre tardan el mismo tiempo en llegar y aunque un destino esté geográficamente más cerca podría tardar más tiempo en llegar 
dado que puede haber pérdida de paquetes o errores en los mismos lo que genera una retransmisión de los mensajes.\\  

Con respecto a los enlaces submarinos los hemos podido detectar mediante los cambios de los $RTT$ promedios. 
Esto era de esperarse dado que recorren miles de kilómetros sin nodos intermedios. Los incidentes los tuvimos al tratar de ubicar 
físicamente los host dado que, como aclaramos anteriormente, las herramientas de geolocalización nos brindan una ubicación aproximada y hasta inexacta.\\

